\chapter*{}
%\thispagestyle{empty}
%\cleardoublepage

%\thispagestyle{empty}

\input{portada/portada_2}



\cleardoublepage
\thispagestyle{empty}

\begin{center}
{\large\bfseries Aplicación Gestión Entrenamientos: Aplicación web con tecnologías modernas}\\
\end{center}
\begin{center}
Antonio de la Vega Jiménez\\
\end{center}

%\vspace{0.7cm}
\noindent{\textbf{Palabras clave}: Web, Node.js, React.js, MongoDB, Integración continua, Despliegue continuo}\\

\vspace{0.7cm}
\noindent{\textbf{Resumen}}\\

Se pretende desarrollar una plataforma web que simplifique el seguimiento de entrenamientos de pesas, proporcionando a los usuarios la información necesaria para llevar un seguimiento sobre la progresión de las cargas a lo largo del tiempo, así como unas estadísticas indicando sobre que grupos musculares se ha hecho mas énfasis y con ello poder evitar estancamientos, descompensaciones y con ello lesiones causadas por una mala progresión.

El objetivo de este desarrollo es crear una plataforma web basada en código libre, que pueda ser usada por cualquier club deportivo así como por cualquier particular, de forma gratuita y con la posibilidad de ser editada para adaptarse a las necesidades de los usuarios.


Para la construcción de este proyecto se hará uso del sistema de control de versiones GIT, el cual nos permite un desarrollo colaborativo y organizado. Además para obtener un software de calidad y fácilmente desplegable, se utilizaran herramientas para:
\begin{itemize}
  \item Tecnologías modernas.
  \item Realizar provisionamiento gestión de la infraestructura.
  \item Validación mediante tests unitarios
  \item Comprobación de la calidad del código y conflictos mediante integración continua.
  \item Actualizaciones rápidas sin interrupciones mediante despliegue continuo.
\end{itemize}

A lo largo del desarrollo se utilizaran ciertas tecnologías para programación elegidas meticulosamente, en el servidor se usara Node.js junto a Express y una base de datos MongoDB, lo cual nos permitirá un desarrollo ágil y nos dará la capacidad de realizar modificaciones sin un coste de trabajo excesivamente alto. En el lado del cliente se usara React.js junto a Redux, que nos permitirá crear una interfaz basada en componentes reutilizables con un estado global, lo cual nos lleva a crear un código mas fácilmente testeable, mantenible y reutilizable. Para apoyar el desarrollo, se hará uso de herramientas como Jest, TravisCI, Ansible y Docker, de las cuales se hablará en detalle posteriormente.

\cleardoublepage


\thispagestyle{empty}


\begin{center}
{\large\bfseries Web App for gym training management: Web App using modern technologies}\\
\end{center}
\begin{center}
Antonio, De la Vega\\
\end{center}

%\vspace{0.7cm}
\noindent{\textbf{Keywords}: Web, Node.js, React.js, MongoDB, Continuous Integration, Continuous Deployment}\\

\vspace{0.7cm}
\noindent{\textbf{Abstract}}\\

The aim is to develop a web platform that simplifies the monitoring of weight training, providing users with the necessary information to track the progression of loads over time, as well as statistics indicating which muscle groups have been made more emphasis and with it to be able to avoid stagnation, decompensation and with it injuries caused by a bad progression.

The objective of this development is to create a web platform based on free code, which can be used by any sports club as well as by any individual, free of charge and with the possibility of being edited to adapt to the needs of users.

\chapter*{}
\thispagestyle{empty}

\noindent\rule[-1ex]{\textwidth}{2pt}\\[4.5ex]

Yo, \textbf{Antonio de la Vega Jiménez}, alumno de la titulación Grado en Ingeniería Informática de la \textbf{Escuela Técnica Superior
de Ingenierías Informática y de Telecomunicación de la Universidad de Granada}, con DNI XXXXXXXXX, autorizo la
ubicación de la siguiente copia de mi Trabajo Fin de Grado en la biblioteca del centro para que pueda ser
consultada por las personas que lo deseen.

\vspace{6cm}

\noindent Fdo: Antonio de la Vega Jiménez

\vspace{2cm}

\begin{flushright}
Granada a 17 de junio de 2018 .
\end{flushright}


\chapter*{}
\thispagestyle{empty}

\noindent\rule[-1ex]{\textwidth}{2pt}\\[4.5ex]

D. \textbf{Juan Julián Merelo Guervós}, Profesor del Departamento de Arquitectura y tecnología de Computadores de la Universidad de Granada.

\vspace{0.5cm}

\textbf{Informan:}

\vspace{0.5cm}

Que el presente trabajo, titulado \textit{\textbf{Aplicación Gestión Entrenamientos: Aplicación web con tecnologías modernas}},
ha sido realizado bajo su supervisión por \textbf{Antonio de la Vega Jimémez}, y autorizamos la defensa de dicho trabajo ante el tribunal
que corresponda.

\vspace{0.5cm}

Y para que conste, expiden y firman el presente informe en Granada a 17 de junio de 2018 .

\vspace{1cm}

\textbf{El director: Juan Julián Merelo Guervós}

\vspace{5cm}

\noindent \textbf{Antonio de la Vega Jiménez \ \ \ \ \ }

\chapter*{Agradecimientos}
\thispagestyle{empty}

       \vspace{1cm}


A todos.

